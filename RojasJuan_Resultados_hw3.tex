%--------------------------------------------------------------------
%--------------------------------------------------------------------
% Formato para los talleres del curso de Métodos Computacionales
% Universidad de los Andes
%--------------------------------------------------------------------
%--------------------------------------------------------------------

\documentclass{article}
\usepackage{amsmath}
\usepackage{amssymb}
\usepackage{graphicx}
\usepackage[utf8]{inputenc}
\usepackage[spanish]{babel}
\usepackage[margin=1.8cm]{geometry}
\usepackage{float}
%\decimalpoint

\title{Resultados Taller \# 3 - Métodos Computacionales}
\author{Juan Diego Rojas}


\begin{document}
	\maketitle
	\section{ODE}
	En esta sección resolvemos numéricamente la ecuación diferencial dada por
	\begin{align*}
		m\ddot{r} = q\dot{r}\times \vec{B}
	\end{align*}
	dónde $r(t) \in \mathbb{R}^3$ determina la posición de una partícula con carga $q$ y masa $m$, bajo la acción de un campo magnético $\vec{B}$. Para esta simulación utilizamos el método de Runge-Kutta. La trayectoria de la partícula desde $t = 0$ a $t = 10s$ es la siguiente:
	\begin{figure}[H]
		\centering
		\includegraphics[width=0.8\linewidth]{ODE.pdf}
		\caption{Trayectoria de la Partícula desde $t=0$ y $t=10s$}
		\label{fig:ODE}
	\end{figure}
	Las siguientes son gráficas adicionales que detallan específicamente algunas de las coordenadas de la trayectoria.
	\begin{figure}[H]
		\centering
		\includegraphics[width=0.8\linewidth]{ODE_additional.pdf}
		\caption{Gráficas adicionales}
		\label{fig:ODE_additional}
	\end{figure}
	\section{PDE}
	El objetivo de esta sección es resolver numéricamente la ecuación diferencial parcial de la membrana de un tambor dada por
	\begin{align*}
		\frac{\partial^2 \Phi(t,x,y)}{\partial t^2} = c^2\left(\frac{\partial^2 \Phi(t,x,y)}{\partial x^2} + \frac{\partial^2 \Phi(t,x,y)}{\partial y^2}\right)
	\end{align*}
	usando el método de diferencias finitas para dos distintas condiciones de frontera. Reduciendo se obtiene la expresión:
	\begin{align*}
		\Phi^{k+1}_{i,j} = \frac{(c\Delta t)^2}{(\Delta y^2)}\left(\Phi_{i+1,j}^{k} - 2\Phi_{i,j}^{k} + \Phi_{i-1,1}^{k}\right) + \frac{(c\Delta t)^2}{(\Delta y^2)}\left(\Phi_{i,j+1}^{k} - 2\Phi_{i,j}^{k} + \Phi_{i,j-1}^{k}\right) + 2\Phi_{i,j}^{k} - \Phi_{i,j}^{k}
	\end{align*}
	donde los índices $i$ indican la posición en $x$, el índice $j$ la posición en $y$ y el índice $k$ la posición temporal. Los parámetros que usamos son $\Delta x = 0.01m = \Delta y$, $c=300 m/s$ en una membrana cuadrada de longitud $L=1m$. Note que para estabilidad de la ecuación diferencial necesitamos que $\Delta t < \frac{\Delta x}{c}$, en nuestro caso usamos $\Delta t =  \frac{\Delta x}{2c}$. En primer lugar, graficamos es estado inicial de la membrana:
	\begin{figure}[H]
		\centering
		\includegraphics[width=0.8\linewidth]{PDE_initial.pdf}
		\caption{Estado inicial de la membrana}
		\label{fig:PDE_initiall}
	\end{figure}
	\subsection{Bordes de la membrana fija}
	En este caso, nuestra condición de frontera es con los bordes de la membrana fija. El estado de la membrana en $t=60ms$ es el siguiente:
	\begin{figure}[H]
		\centering
		\includegraphics[width=0.8\linewidth]{PDE_fixed.pdf}
		\caption{Estado de la membrana en $t=60ms$}
		\label{fig:pdefixed}
	\end{figure}
	
\end{document}
